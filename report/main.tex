\documentclass[a4paper,table]{article}

\usepackage[frenchb]{babel}
\usepackage[utf8]{inputenc}
\usepackage[T1]{fontenc}
\usepackage{amsmath}
\usepackage{graphicx}
\usepackage[colorinlistoftodos]{todonotes}
\usepackage{a4wide}
\usepackage{enumitem}

\usepackage{xcolor}
\definecolor{dkgreen}{rgb}{0,0.6,0}
\definecolor{steelblue}{rgb}{0.16,0.37,0.58}
\definecolor{gray}{rgb}{0.5,0.5,0.5}
\definecolor{mauve}{rgb}{0.58,0,0.82}
\definecolor{blue}{rgb}{0,0,0.7}
\definecolor{hlColor}{rgb}{0.94,0.94,0.94}
\definecolor{shadecolor}{rgb}{0.96,0.96,0.96}
\definecolor{TFFrameColor}{rgb}{0.96,0.96,0.96}
\definecolor{TFTitleColor}{rgb}{0.00,0.00,0.00}
\definecolor{lightred}{rgb}{1,0.96,0.96}
\definecolor{darkred}{rgb}{0.85,0.33,0.31}
\definecolor{lightblue}{HTML}{EBF5FA}
\definecolor{lightblue2}{HTML}{E3F2FA}
\definecolor{darkblue}{HTML}{D2DCE1}
\definecolor{lightyellow}{HTML}{FFFAE6}
\definecolor{darkyellow}{HTML}{FAE6BE}

\usepackage{listings}
\lstset{
	language=C,
	basicstyle=\scriptsize,
	numbers=left,                   % where to put the line-numbers
  	numberstyle=\tiny\color{gray},
	commentstyle=\color{steelblue},
	stringstyle=\color{darkred},
	backgroundcolor=\color{shadecolor},
    keywordstyle=\color{dkgreen},
	frame=single,                   % adds a frame around the code
 	rulecolor=\color{black},
	emph={},
	emphstyle=\color{mauve},
	morekeywords=[2]{},
	keywordstyle=[2]{\color{dkgreen}},
	showstringspaces=false,
  	tabsize=4,
	moredelim=[is][\small\ttfamily]{/!}{!/},
	breaklines=true
}

\usepackage{hyperref}
\hypersetup{
	colorlinks=true, % false: boxed links; true: colored links
	linkcolor=black, % color of internal links
	urlcolor=blue,   % color of external links
	citecolor=blue
}
\newcommand{\hhref}[1]{\href{#1}{#1}}

\usepackage{makecell}

\usepackage{eurosym} %\euro -> €

\usepackage{soul}
\sethlcolor{hlColor}

\title{INF5101A - TP2 MPI}

\date{\today}

\begin{document}
\maketitle
\newpage
\tableofcontents
\newpage

\section{Introduction}

%TODO

\section{Calcul de $\pi$}

Dans cet exercice, l'objectif est de calculer une valeur la plus proche de
$\pi$. Cette valeur peut s'obtenir avec un calcul d'intégrale. Par conséquent,
pour atteindre la plus grande précision, il faudra effectuer les calculs avec
des très petits pas entre les bornes de l'intervalle. En procédant de cette
façon, le calcul de la valeur de $\pi$ demandera beaucoup de temps. A l'aide de
la bibliothèque MPI, nous pouvons paralléliser ces calculs sur un nombre de
processeurs que nous choisissons. Il est ainsi possible d'accélérer le calcul
de $\pi$. \\

Pour implémenter cette théorie, nous avons fait en sorte que chaque processeur
qui exécute le programme se détermine des bornes d'intervale en fonction de
leur numéro de tâche. Lorsque tous les calculs sont terminés, nous utilisons la
fonction \hl{MPI\_Reduce()} qui est ici configurée pour calculer la somme de
toutes les valeurs obtenues par chaque processeur.

\end{document}
